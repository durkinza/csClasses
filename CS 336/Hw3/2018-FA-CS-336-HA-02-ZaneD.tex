\documentclass[14pt]{article}

\usepackage{enumitem}
\usepackage{calc}
\usepackage{tikz}
\usetikzlibrary{trees}
\parskip=.9ex
\textwidth=7.0in
\textheight=9.0in
\oddsidemargin=-.25in
\topmargin=-.75in

\begin{document}

\title{Selected Problems from Stallings Textbook: Chapter 01}

\author{Zane Durkin\\
    University of Idaho}

\begin{description}[leftmargin=!, labelwidth=\widthof{\bfseries Author(s) Name(s)}]
\item [Year and Semester] 2018 FALL
\item [Course Number] CS-336
\item [Course Title] Intro. to Information Assurance
\item [Work Number] HA-03
\item [Work Name] 
\item [Work Version] Version 1
\item [Long Date] Wednesday, 5 September 2018
\item [Author(s) Name(s)] Zane Durkin
\end{description}


\begin{abstract}
In this article I will be going over 6 Laws or statues. Two of these laws will be ones passed by two different states. Three of them will be ones that have been passed by the national government, and the last law will be one that has been passed by a different country's government.
\end{abstract}

\section{Idaho Statute 18-2202}
\begin{description}[leftmargin=!, labelwidth=\widthof{\bfseries Year Approved}]
    \item [Name] Title 18 (Crimes and Punishments), Chapter 22 (Computer Crime), Statute 18-2202
    \item [Abbreviation] 18-2202
    \item [Year Approved] 1984
    \item[Description] Idaho law 18-2202 defines the actions that are required to match that of a computer crime. The law defines that you are not allowed to knowingly access, alter, or damage any computer or computer system. The limitations of this law are that you know that what you are doing, and that you don't have authorization to do it. This law explains in detail that the act must be done for the purpose of: Obtaining money, property, or services. An act that is done for another purpose does not match this description and would therefor fall under another law \cite{Idaho182202}.
\end{description}


\newpage
\bibliographystyle{ACM}
\bibliography{../../citations}

\end{document}
