\documentclass[14pt]{article}

\usepackage{enumitem}
\usepackage{calc}
\usepackage{tikz}
\usetikzlibrary{trees}
\parskip=.9ex
\textwidth=7.0in
\textheight=9.0in
\oddsidemargin=-.25in
\topmargin=-.75in

\begin{document}

\title{Selected Problems from Stallings Textbook: Chapter 01}

\author{Zane Durkin\\
    University of Idaho}

\begin{description}[leftmargin=!, labelwidth=\widthof{\bfseries Author(s) Name(s)}]
\item [Year and Semester] 2018 FALL
\item [Course Number] CS-336
\item [Course Title] Intro. to Information Assurance
\item [Work Number] HA-02
\item [Work Name] Cybersecurity and The Law
\item [Work Version] Version 1
\item [Long Date] Wednesday, 5 September 2018
\item [Author(s) Name(s)] Zane Durkin
\end{description}


\begin{abstract}
In this article I will be going over 6 Laws or statues. Two of these laws will be ones passed by two different states. Three of them will be ones that have been passed by the national government, and the last law will be one that has been passed by a different country's government.
\end{abstract}

\section{Idaho Statute 18-2202}
\begin{description}[leftmargin=!, labelwidth=\widthof{\bfseries Year Approved}]
    \item [Name] Title 18 (Crimes and Punishments), Chapter 22 (Computer Crime), Statute 18-2202
    \item [Abbreviation] 18-2202
    \item [Year Approved] 1984
    \item[Description] Idaho law 18-2202 defines the actions that are required to match that of a computer crime. The law defines that you are not allowed to knowingly access, alter, or damage any computer or computer system. The limitations of this law are that you know that what you are doing, and that you don't have authorization to do it. This law explains in detail that the act must be done for the purpose of: Obtaining money, property, or services. An act that is done for another purpose does not match this description and would therefor fall under another law \cite{Idaho182202}.
\end{description}

\section{Utah Law 76-6-703}
    \begin{description}[leftmargin=!, labelwidth=\widthof{\bfseries Year Approved}]
        \item [Name] Computer crimes and penalties -- interfering with critical infrastructure.
        \item [Abbreviation] 76-6-703
        \item [Year Approved] 2017
        \item [Description] Utah Computer crimes act 76-6-703 describes what actions are considered to be unlawful. This law, along with being much more descriptive than the Idaho law, covers many more acts such as the transmission or disclosure of data or technology. The Utah law expressly criminalizes alteration, destruction, copy, transmission, discovery, or disclosure of computer technology when the user knowingly commits the acts and is not authorized to do so. The law also states that if a person uses a computer service to prefer the act, and the computer service did not knowingly assist the person to commit the act, then the computer service is not guilty of violating the law \cite{Utah766703}.
    \end{description}

\section{Federal Laws}
\subsection{Digital Millennium Copyright Act}
    \begin{description}[leftmargin=!, labelwidth=\widthof{\bfseries Year Approved}]
        \item [Name] Digital Millennium Copyright Act
        \item [Abbreviation] DMCA
        \item [Year Approved] 1998
        \item [Description]  The Digital Millennium Copyright Act addresssed a few major copright topics, along with implementing a few other treaties that were created by the World Intellectual Property Organization (WIPO). The DMCA requires that before a lawsuit can be created by a copyright owner, the owner must first register the copyright with the copyright office. The owner of the copyright is not required to register the copyright with the copyright office, but they can choose to do so for additional legal benefits such as the ability to initialize law suits. The law prevents the copying of material for sale, and allows for appropriate security of the copyrighted materials. There are exceptions on tehe copyright act that permits the circumvention of access control measures for the purpose of testing the security. \cite{DMCA}.
    \end{description}
\subsection{Computer Fraud and Abuse Act}
    \begin{description}[leftmargin=!, labelwidth=\widthof{\bfseries Year Approved}]
        \item [Name] Computer Fraud and Abuse Act
        \item [Abbreviation] CFAA
        \item [Year Approved] 2
        \item [Description]The Computer Fraud and Abuse Act was created to protect computers that are used by the inisted States Government, or by a financial institution. The Act prohibits interference of communication to any person entitled to recieve the transmition. The law prohibits accessing a "protected" computer without authorization or in a way that exceeds the authorized access.  \cite{CFAA}
    \end{description}
\subsection{Can Spam Act}
    \begin{description}[leftmargin=!, labelwidth=\widthof{\bfseries Year Approved}]
        \item [Name] Controlling the Assault of Non-Solicited Pornography and Marketing Act
        \item [Abbreviation] CAN-SPAM
        \item [Year Approved] 2008
        \item [Description] The CAN SPAM Act is a federal regulation on commercial mail messages. This Act was originally created to control commercial mail messages and the assault of non-solicited pornography. The Act required that all commercial mail that is not a direct response to a transaction or event on the user's account must contain a way to easily un-subscribe from further messages. The Act also prohibits charging a fee or requiring more information to unsubscribe from the messaging list. The Act also creates a requirement for all commercial mail that contains explicit content to not that the message body contains the content in the subject line of the message.\cite{CANSPAM}.
    \end{description}
    
\section{Other Country law}
    \begin{description}[leftmargin=!, labelwidth=\widthof{\bfseries Year Approved}]
        \item [Name] C
        \item [Abbreviation] 7
        \item [Year Approved] 2
        \item [Description] \cite{GDPR}.
    \end{description}

\newpage
\bibliographystyle{ACM}
\bibliography{../../citations}

\end{document}
