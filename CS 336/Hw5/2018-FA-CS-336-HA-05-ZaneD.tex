\documentclass[14pt]{article}

\usepackage{enumitem}
\usepackage{calc}
\usepackage{tikz}
\usetikzlibrary{trees}
\parskip=.9ex
\textwidth=7.0in
\textheight=9.0in
\oddsidemargin=-.25in
\topmargin=-.75in

\begin{document}

\title{Asymmetric Encryption: RSA}

\author{Zane Durkin\\
    University of Idaho}
\begin{description}[leftmargin=!, labelwidth=\widthof{\bfseries Author(s) Name(s)}]
\item [Year and Semester] 2018 FALL
\item [Course Number] CS-336
\item [Course Title] Intro. to Information Assurance
\item [Work Number] HA-05
\item [Work Name] Asymmetric Encryption: RSA
\item [Work Version] Version 1
\item [Long Date] Tuesday, 2 October 2018
\item [Author(s) Name(s)] Zane Durkin
\end{description}
\begin{abstract}
In this article, since I do not have a partner, I will be completing both sides of the RSA Lab, along with the problems from the textbook and the review questions.
\end{abstract}

\section{RSA Lab}
I will be using two prime numbers, $p$=227 and $q$=191.
Here are the modulus and Euler totient, Modulus: 43357, Euler Totient:42940
Using these values I can generate a public and private key pair.
For my public key, I will pick a value $e$ such that it has no common factors (except 1) with the Euler totient
My $e$ is: 27
For my private key, I will find a value $d$ such that $d*e mod $Euler Totient = 1
My private key is: 12723

Since I do not have a partner, I will encrypt a message with my own public key, and then decrypt it with my private key.
\subsection{Encryption}
My message to encrypt will be "HELLOHELLOHELLO"
The conversion process is as follows:

\begin{center}
    \begin{tabular}{| l | l | l | l |}
      \hline
      \multicolumn{2}{|c|}{Plain Text} &
      \multicolumn{2}{|c|}{Cipher Text} \\
      \hline
      Trigraph &
      Trigraph code &
      Enciphered Code &
	  Quadragraph \\
      \hline
      HEL & 4847 & 5611 & AIHV\\
	  LOH & 7807 & 33195 & BXCT\\
      ELL & 3001 & 11387 & AQVZ\\
      OHE & 9650 & 31569 & BUSF\\
      LLO & 7736 & 33322 & BXHQ\\
      \hline 
    \end{tabular}
\end{center}
The cipher text would then be:
AIHVBXCTAQVZBUSFBXHQ
\subsection{decryption}
Now to decrypt the string using my private key.
I will be decoding the same string since I do not have a partner to work with. 
\begin{center}
    \begin{tabular}{| l | l | l | l |}
      \hline
      \multicolumn{2}{|c|}{Cipher Text} &
      \multicolumn{2}{|c|}{Plain Text} \\
      \hline
	  Quadragraph &
      Quadragraph Code &
      Deciphered code &
      Deciphered Trigraph &
      \hline
      AIHV & 5611  & 4847 & HEL\\
      BXCT & 33195 & 7807 & LOH\\
      AQVZ & 11387 & 3001 & ELL\\
      BUSF & 31569 & 9650 & OHE\\
      BXHQ & 33322 & 7736 & LLO\\
      \hline
    \end{tabular}
\end{center}
So the deciphered text would be:
HELLOHELLOHELLO
which matches the original message

\subsection{RSA Questions}
\begin{itemize}
  \item Private encryption is a one way encryption that prevents third parties from decoding a message after it has been encrypted. The encoded message can only be decoded by the owner of the private key. Symmetric encryption can be encrypted and decrypted using the same key, so either party is able to create and read messages. 
  \item The benefit of private encryption is that the public key does not need to be shared prior to the communication to ensure confidentiality. Symmetric encryption has the benefit of a much smaller load to encrypt  

\end{itemize}

\newpage


\section{Problem 21.6 from the textbook}
Parts a and c only.
\newpage

\section{Review Questions}
21.1 to 21.5

\newpage

\bibliographystyle{ACM}
\bibliography{../../citations}

\end{document}
