\documentclass[14pt]{article}

\usepackage{enumitem}
\usepackage{calc}
\usepackage{tikz}
\usetikzlibrary{trees}
\parskip=.9ex
\textwidth=7.0in
\textheight=9.0in
\oddsidemargin=-.25in
\topmargin=-.75in

\begin{document}

\title{Asymmetric Encryption: RSA}

\author{Zane Durkin\\
    University of Idaho}
\begin{description}[leftmargin=!, labelwidth=\widthof{\bfseries Author(s) Name(s)}]
\item [Year and Semester] 2018 FALL
\item [Course Number] CS-336
\item [Course Title] Intro. to Information Assurance
\item [Work Number] HA-05
\item [Work Name] Asymmetric Encryption: RSA
\item [Work Version] Version 1
\item [Long Date] Tuesday, 2 October 2018
\item [Author(s) Name(s)] Zane Durkin
\end{description}
\begin{abstract}
In this article, since I do not have a partner, I will be completing both sides of the RSA Lab, along with the problems from the textbook and the review questions.
\end{abstract}

\section{RSA Lab}
I will be using two prime numbers, $p$=227 and $q$=191.\\
Here are the modulus and Euler totient, Modulus: 43357, Euler Totient:42940\\
Using these values I can generate a public and private key pair.\\
For my public key, I will pick a value $e$ such that it has no common factors (except 1) with the Euler totient\\
My $e$ is: 27\\
For my private key, I will find a value $d$ such that $d*e $ mod Euler Totient = 1\\
My private key is: 12723\\
Since I do not have a partner, I will encrypt a message with my own public key, and then decrypt it with my private key.
\subsection{Encryption}
My message to encrypt will be HELLOHELLOHELLO\\
The conversion process is as follows:

\begin{center}
    \begin{tabular}{| l | l | l | l |}
      \hline
      \multicolumn{2}{|c|}{Plain Text} &
      \multicolumn{2}{|c|}{Cipher Text} \\
      \hline
      Trigraph &
      Trigraph code &
      Enciphered Code &
      Quadragraph \\
      \hline
      HEL & 4847 & 5611 & AIHV\\
      LOH & 7807 & 33195 & BXCT\\
      ELL & 3001 & 11387 & AQVZ\\
      OHE & 9650 & 31569 & BUSF\\
      LLO & 7736 & 33322 & BXHQ\\
      \hline
    \end{tabular}
\end{center}
The cipher text would then be:
AIHVBXCTAQVZBUSFBXHQ
\subsection{decryption}
Now to decrypt the string using my private key.
I will be decoding the same string since I do not have a partner to work with.
\begin{center}
    \begin{tabular}{| l | l | l | l |}
      \hline
      \multicolumn{2}{|c|}{Cipher Text} &
      \multicolumn{2}{|c|}{Plain Text} \\
      \hline
      Quadragraph &
      Quadragraph Code &
      Deciphered code &
      Deciphered Trigraph \\
      \hline
      AIHV & 5611  & 4847 & HEL\\
      BXCT & 33195 & 7807 & LOH\\
      AQVZ & 11387 & 3001 & ELL\\
      BUSF & 31569 & 9650 & OHE\\
      BXHQ & 33322 & 7736 & LLO\\
      \hline
    \end{tabular}
\end{center}
So the deciphered text would be:
HELLOHELLOHELLO\\
which matches the original message.

\subsection{RSA Questions}
\begin{itemize}
  \item Private encryption is a one way encryption that prevents third parties from decoding a message after it has been encrypted. The encoded message can only be decoded by the owner of the private key. Symmetric encryption can be encrypted and decrypted using the same key, so either party is able to create and read messages.
  \item The benefit of private encryption is that the public key does not need to be shared prior to the communication to ensure confidentiality. Symmetric encryption has the benefit of taking much less effort to encrypt and decrypt from a processing standpoint.
  \item RSA is an acronym of the creators of the algorithm: Ron Rivest, Adi Shamir and Leonard Adelman.
  \item RSA is very useful for sending encrypted messages across a network with confidentiality ensured.
  \item Creating a public key requires choosing two prime numbers, and a number that is relatively prime to the phi(n) of the two prime numbers.
  \item The numbers to be shared are the n value (p*q) and the e value (The number that is relatively prime to n).
  \item encrypting a message uses the formula $m^{e} \% n = c$ where $m$ is a message and $c$ is the cipher text
    Decrypting a message uses a similar formula $c^{d} \% n = m$.
  \item RSA by it's self do not make a great use of diffusion, since each bit of the message would normally be encrypted independently, but the use of trigraph codes creates a better use of diffusion since each bit of the message will effect a larger portion of the cipher text.
  \item RSA provides good use of confusion, since the value of the key effects each bit independently, the key has a dramatic effect on the outputted ciphertext.

\end{itemize}

\newpage


\section{Problem 21.6 from the textbook \cite{stallings}}
Parts a and c only.
\begin{enumerate}[label={\alph*)}]
  \item $p=3$\\$q=11$\\$e=7$\\$n=33$\\$phi(n)=20$\\$d=3$\\$m=5$\\
  \begin{description}[leftmargin=!, labelwidth=\widthof{\bfseries Decrypting}]
    \item [Encrypting] $5^{7} \% 33 = c$\\
    So the Ciphertext would be $c = 14$
    \item[Decrypting] $14^{3} \% 33 = m$\\
    So the Plaintext would be $m = 5$
  \end{description}
  \setcounter{enumi}{2}
  \item $p=7$\\$q=11$\\$e=17$\\$n=77$\\$phi(n)=60$\\$d=53$\\$m=8$\\
  \begin{description}[leftmargin=!, labelwidth=\widthof{\bfseries Decrypting}]
    \item [Encrypting] $8^{17} \% 77 = c$\\
    So the Ciphertext would be $c = 57$
    \item[Decrypting] $57^{53} \% 77 = m$\\
    So the Plaintext would be $m = 8$
  \end{description}
\end{enumerate}
\newpage

\section{Review Questions}
\subsection{Review Question 21.1}
  hash functions compress data to fit a set size of bits in their output. The compression is done by hashing the data in blocks of a set size.
\subsection{Review Question 21.2}
  SHA requres 5 steps of operations:
  \begin{itemize}
  \item Appending padding buts to the message,
  \item Appending the original length of the message to the end of the message.
  \item Initializing the hash buffer with preset values from fractional parts of the first eight prime numbers
  \item Process message in 1024-bit blocks
  \item output the final hash of the message
  \end{itemize}
\subsection{Review Question 21.3}
HMAC is designed to allow for easy replacement of embedded hash functions. The replacement of a hash function is done without any modification to HMAC
\subsection{Review Question 21.4}
A one-way function is one that is either impossible or really hard to reverse. So finding the original value given the output value would be nearly or entirely impossible.
\subsection{Review Question 21.5}
Diffie-Hellman's key exchange is done by requiring each party to generate a part of a key, then using two primes and a primitive root to encrypt these keys for transfer across an open network.
\newpage

\bibliographystyle{ACM}
\bibliography{../../citations}

\end{document}
