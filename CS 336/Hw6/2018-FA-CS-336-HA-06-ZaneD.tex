\documentclass[14pt]{article}

\usepackage{enumitem}
\usepackage{calc}
\usepackage{tikz}
\usetikzlibrary{trees}
\parskip=.9ex
\textwidth=7.0in
\textheight=9.0in
\oddsidemargin=-.25in
\topmargin=-.75in

\begin{document}

\title{Argument corresponding to the problems indicated and described}

\author{Zane Durkin\\
    University of Idaho}
\begin{description}[leftmargin=!, labelwidth=\widthof{\bfseries Author(s) Name(s)}]
\item [Year and Semester] 2018 FALL
\item [Course Number] CS-336
\item [Course Title] Intro. to Information Assurance
\item [Work Number] HA-06
\item [Work Name] Argument corresponding to the problems indicated and described
\item [Work Version] Version 1
\item [Long Date] Sunday, 7 October 2018
\item [Author(s) Name(s)] Zane Durkin
\end{description}
\begin{abstract}
In this article I will be giving my arguments for the corresponding problems as indicated below.
\end{abstract}

\section{Problems 19.09 through 19.12 from the textbook \cite{stallings}}
\subsection{Problem 19.09}
In the scenario given, it is apparent that the cracking of the user's passwords was not allowed within the company and thus should not have been attempted. The proper route to strengthening passwords would have been to set passwords requirements (i.e. must contain a capital and lower case letter, a symbol, and be longer than 8 characters) and require that everyone create new passwords that matched these standards. If the cracking of passwords was required for some reason, the employee should have requested permission from his superiors and the user's whose accounts would be tested. As shown in the firgure 19.6 \cite{stallings}[p.~635], personal data should be protected from unauthorized access.
\newpage
\subsection{Problem 19.10}
{\renewcommand{\arraystretch}{2}%
  \begin{tabular}{|l|l|l|l|}
    \cline{2-4}
    \multicolumn{1}{r|}{}& ACM & IEEE & AITP \\ \hline
    Dignity and worth of people & 1.1, 1.4 , 1.6, 1.7, 2.1, 34 & 1, 8 & 6, 11, \\ \hline
    Personal integrity & 1.3, 1.5, 1.7, 2.1, 2.6, 2.8, 4.1 & 1, 2, 4 & 1, 2, 3, 4, 7, 11, 12 \\ \hline
    Responsibility for work & 1.7, 2.2, 2.3, 2.5, 2.6, 2.8, 4.2 & 3, 4, 6 & 1, 2, 4, 8 \\ \hline
    Confidentiality of information & 1.3, 1.7, 1.8, 2.8, 3.3 & 3 & 4 \\ \hline
    Public safety & 1.1, 1.2, 2.7, 3.4, 3.5 & 1, 2 & 6, 12 \\ \hline
    Health & 1.1, 1.2 & 1, 9 & 12 \\ \hline
    Welfare & 1.1, 2.7, 3.2 & 1, 9 & 9, 12 \\ \hline
    Participation in professional societies & 1.6, 2.1, 2.2, 2.3, 2.5, 2.6 & 7, 10 & 1, 2, 7, 10, 11 \\ \hline
    Knowledge about technology related to social power & 3.1, 3.2 & 6 & 8 \\ \hline
  \end{tabular}}
\newpage
\subsection{Problem 19.11}
The 1982 version of the ACM code of Ethics has changed slightly since the 1997 version of the code.\\
Elements found in the 1982, but not in the 1997 version:\\
	\begin{itemize}
		\item 
	\end{itemize}
\newpage
\subsection{Problem 19.12}

\newpage

\section{Problems}
Develop 3 non-trivial example scenarios in which you would need to apply ethical thinking in order to come up with the most ethical course of action. Then indicate all the current ACM Code of Ethics and Professional Conduct statements that you believe would apply to each scenario and explain your rationale.
\subsection{Scenario 1}

\newpage
\subsection{Scenario 2}

\newpage
\subsection{Scenario 3}


\newpage
\bibliographystyle{ACM}
\bibliography{../../citations}
\end{document}
